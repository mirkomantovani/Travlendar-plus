The set of functionalities we actually implemented with respect to the ones we initially defined are:

\subsection{[F1] Signup and Login}
Travlendar+ users must sign up the first time they intend to use the App, further usages of the app will require a login to access all its functionalities.

\subsection{[F2] Meeting creation}
This is the most important function of the app, it allows to generate an event related to an appointment. It requires the user to define all the details such as date, time, location.

\subsection{[F3] Preferences set up}
An important feature of Travlendar+ consists in allowing the user to filter out specific routes depending on some constraints about the travel, or to set break-dedicated time slots. In particular the preferences we implemented are:

\begin{itemize}
\item Minimize carbon footprint option, the result of flagging this option is that if possible (there exist at least a route and the travel means constraints are satisfied) only walking, cycling and public transportations routes will be taken into consideration.
\item Avoid tolls option which will only affect driving mode, only routes without tolls will be considered.
\item Avoid motorways option which will only affect driving mode, only routes without highways (motorways) will be considered.
\item Setting up a maximum walking distance and taking into consideration only the routes which satisfy this constraint
\item Setting up a maximum cycling distance and taking into consideration only the routes which satisfy this constraint
\item Setting up a time from which the routes involving public transportations will be discarded, in particular, the User can select the time between the range 18:00 and 5:30, from that time till the morning any route involving public transportations will be avoided.
\item Specifying the travel means the user intends to use for his travels, possible travel means are: \begin{itemize}
\item Owned car
\item Shared car
\item Owned bike
\item Shared bike
\item Walking
\item Public transports
\end{itemize}
\end{itemize}

\subsection{[F4] Warnings management}
In case a warning is generated by the system due to a possible overlap among two or more meetings, the user must solve the warning. In other words, the user has to decide whether he wants to ignore the warning notification or he intends to modify some meetings to be sure that he can reach and participate to all his appointments.
In particular the warning are created if... blah blah

\subsection{[F5] Route generation}
When requested, the app is able to find a route, if possible, from the specific location the user is at in the moment to the location of the considered meeting, the suggested travel will fit and satisfy the selected preferences.

\subsection{[F8] Update/Delete meeting}
This function is both basic and relevant, it allows the user to customize his meetings after their creation. In particular, everything except the name (since it's part of the identification of the meeting) can be modified.

\subsection{[F9] Add/Delete break}

This feature at first was specified as part of the preferences, however, since we considered that this is one of the main features, we named it alone as an independent functionality. The Flexible break is a slot of time in the day to be reserved for specific purposes (such as a lunch for example), the break has a starting time and an ending time, which are the extremes times in which the break must be contained, the break itself however will only last the time specified by the Duration attribute. The day to specify for the break is in the form of the day of week (monday, tuesday, ecc.), which means the first occurrence of that day of week from the creation time. The Break could also be recurrent, which means every week in that day of the week a break will be scheduled. The break is flexible in the sense that the actual break only lasts for a time specified by the duration and Travlendar will make sure that the user has at least that free time available inbetween his meetings, if not a Warning will be generated.



 
