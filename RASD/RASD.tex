\documentclass{article}
\usepackage{float}
\usepackage{textcomp}
\usepackage{graphicx}
\usepackage{booktabs}
\usepackage{color}
\usepackage{verbatim}
\usepackage{listings}
\usepackage{underscore}
%\usepackage{float} % here for H placement parameter
\usepackage{flafter} 
\setcounter{secnumdepth}{5}
\usepackage[bookmarks=true]{hyperref}
\author{Mirko Mantovani (893784), Matteo Marziali (893904)} 
\date{\today}
\title{Politecnico di Milano
	\\A.A. 2017\@-\@2018
	\\Software Engineering II project \\ \textbf{Travlendar+}
	\\\textbf{R}equirements \textbf{A}nalysis \\and\\ \textbf{S}pecifications \textbf{D}ocument}
	\hypersetup{pdftitle={Software Requirement Specification},    % title
	pdfauthor={Mirko Mantovani, Matteo Marziali},                     % author
	pdfsubject={RASD},                        % subject of the document
	colorlinks=true,       % false: boxed links; true: colored links
	linkcolor=black,       % color of internal links
	citecolor=blue,       % color of links to bibliography
	filecolor=black,        % color of file links
	urlcolor=purple,        % color of external links
}
\begin{document}
\maketitle
\begin{center}
	\includegraphics[width=5cm]{polimi-logo}
\end{center}
\clearpage
\tableofcontents
\clearpage

\section{Introduction}

\subsection{Purpose}
This document is intended to provide information about how the implementation was actually accomplished, what we really implemented in term of functionalities and requirements, as well as how the testing phase was done in detail, the test cases we performed and their outcome. It also includes different installation instructions possibilities with the main solutions to struggles and problems you may encounter during this phase.


\newpage
\subsection{Scope}
Even if we are very confident about the success of our idea, initially, Travlendar+ will have a restricted domain, indeed we will experiment it only in the Italian city of Milan. 
In order to provide the most complete assistance, Travlender+ will suggest different paths and a wide range of transports such as bike, even shared, your car or a shared one, taxi, bus, and also.. your feet!
It goes without saying that to organize this kind of service in the most valuable way we must interact with a huge number of local institutions which provide different services in town. 


\subsection{Goals}


\subsection{Definition and Acronyms}

\subsubsection{Definitions}
\begin{itemize}

\item  \textbf{App:} this is the abbreviation for application, in particular this term is used meaning a mobile application.

\item  Delay notification function: this phrase refers to the function which allows to notify the participants of a meeting through an email in case the user is late.

\item  \textbf{Travel:} a travel is any suggested path that goes from the starting point to the meeting location.

\item  \textbf{Route:} this term is used as a synonym of travel.

\item  \textbf{Warning:} warning is the word used to define the conflict between two meetings.

\item  \textbf{Conflict:} a conflict between two or more meetings is what enables the creation of a warning, it means that the set of meetings in conflict are scheduled too close in time in order for the user to be able to attend them all in time.

\item  \textbf{Calendar:} the calendar contains the list of meetings and is grouped by day.

\item  \textbf{Meeting:} is an important keyword of the application, it includes all the informations of an appointment.

\item  \textbf{Reminder:} a reminder is a sort of an alarm triggered at a certain time before an appointment is starting.


\end{itemize}

\subsubsection{Acronyms}
\begin{itemize}
	\item ETA:\@ estimated time of arrival: the time, estimated by the system, that the closest available taxi will take to get to the starting location of the ride.
	\item API:\@ application programming interface; it is a set of routines, protocols, and tools for building software applications on top of this one.
	
\subsection{Revision}
\begin{itemize}
\item \textbf{V2 on \date{\today}:} Complete revision of component diagrams and database after implementation. Added new user interface screens on web application.
\end{itemize}

	
\end{itemize}
\subsection{Actors}
\begin{itemize}
	\item \textit{Guest}: 
	\item \textit{Logged in users}:

\end{itemize}

\subsection{References}
\begin{itemize}
	\item The document with the assignment for the project
	\item The IEEE Standard for SRS 
\end{itemize}
\subsection{Document Structure}
This document is structured in seven sections, here is an overview of the contents of each and every one:
\begin{itemize}
	\item \textbf{Introduction}: 
This section provides a general introduction and overview of the Design Document and the covered topics that were not previously taken into account by the RASD. 
	\item \textbf{Architectural Design}:
	This section shows the main system components together with sub-components and their relationship. 
	This section is divided into different parts whose focus is mainly on design choices, interactions, used architectural styles and patterns.  
	\item \textbf{Algorithm Design}: 
This section provides a high-level description and details about some of the most crucial and critical algorithms to be implemented in the system-to-be. 
	\item \textbf{User Interface Design}: 
 It provides an overview on how the user interface will look like and behave giving a more complete view with respect to those contained in the RASD. 
	\item \textbf{Requirements traceability}: 
This section describes how the requirements defined in the RASD are mapped and are satisfied by the design elements and components defined in this document. 
	\item \textbf{Implementation, integration and test plan}: 
This section is used to explain the strategies for implementations and testing that will be adopted in the development part of the project.
\item \textbf{Appendix}: 
Here we provide information about the used software and the effort spent to redact this document. 
\end{itemize}

\clearpage
\section{Overall description}

\subsection{Product perspective}
Our idea is to create a personal companion application to help users managing and organizing their daily life. According to this intention, we would like to realize an extremely friendly user interface and a lightweight software in order to make Travlendar+ affordable to many people and runnable by many devices.\\
In order to make use of every functionality the devices require GPS service and an internet connections for most of the services.\\
\\
Since Travlendar+ is going to offer many routes depending on different travel means, it will necessarily have to interact with many institutions such as public transport and car/bike sharing providers. This aspect will affect both the software and the hardware design. Indeed, it is necessary to query data about the shared cars, bikes and the taxis location around the city and to retrieve information about trains and buses schedules. \\
Hence, our system must be very fast and dynamic to support a huge number of query in a few seconds, moreover to interview external databases it's strictly required that the users have an active internet connection. \\
\\
Concerning the hardware, we intend to have a database which only contains username and password of all our customers. For sure, it seems useless having a database for those kinds of data but in our idea this choice allows the app to be updated and improved easily in the future, for instance saving on the database clients' routes and meetings. \\




\clearpage

\subsection{Product functionalities}

\begin{itemize}

\item \textbf{[\hypertarget{F1}{F1}] Signup and Login}: \\Travlendar+ users must sign up the first time they intend to create a meeting and further usages of the app will require a login to access all its functionalities.
\item \textbf{[\hypertarget{F2}{F2}] Meeting creation}: \\This is the most important function of the app, it allows to generate an event related to an appointment. It requires the user to define all the details such as date, time, location, starting point, preferences etc. 
\item \textbf{[\hypertarget{F3}{F3}] Preferences set up}: \\An important feature of Travlendar+ consists in allowing the user to filter out specific routes depending on some constraints about the travel, or to set break-dedicated time slots.
\item \textbf{[\hypertarget{F4}{F4}] Delays management}:  \\If the app had noticed, according to the estimated travel time, that the user is in late, and he previously had inserted the email address of the meeting’s participants, Travlendar+ would notify them about the delay. 
\item \textbf{[\hypertarget{F5}{F5}] Route generation}:\\ the main hidden function of Travlendar+ is to automatically compute and suggest to the user the best travel among those which fit the preferences he has selected.
\item \textbf{[\hypertarget{F6}{F6}] Reminder management}: The applications allows users to set up reminders for a certain Meeting in order for the user not to be late for it.
\item \textbf{[\hypertarget{F7}{F7}] Recurrent events management}:\\ The smartest function Travlendar+ will offer; it consists in allowing the user to select events to be rescheduled periodically just creating one meeting. Done this choice, the app. Automatically manages to  reschedule the specific meeting according to the period that the user establish, for instance one week, one month.
\item \textbf{[\hypertarget{F8}{F8}] Update meetings}: \\This function is both basic and relevant, it allows the user to customize his meetings after their creation. In other words, through this functionality the user can modify each one meeting details, even in case of a warning is generated.
\item \textbf{[\hypertarget{F9}{F9}] Warnings management}: \\ In case a warning is generated by the system due to a possible overlap among two or more meetings, the user must solve the warning. In other words, the user has to decide wheter he wants to ignore the overlap notification or he intend to modify some meetings to be sure that he can reach and participate to all his appointments.


\end{itemize}

 


\subsection{User characteristics}
According to our idea, Travlendar+ does not have a specific customers range, it is supposed to be used by both male and female, whatever their age is.
Obviously, considering that we intend to produce a mobile application, people who want to use Travlendar+ should be familiar with a portable device like a smartphone or a tablet. 
For sure, users should respect several requirements due to the travel means they are going to take. For instance, when cars are selected as active in the means of transport list, the user is supposed to have a valid driver licence, taking the bus is allowed only with an appropriate ticket.
Moreover, users interested in dealing with Travlendar+ services must have an e-mail address, primarily due to register and authenticate themselves, secondly to use the delay notification function.



\subsection{Assumptions and dependencies}

\subsubsection{Domain Assumptions}
\begin{itemize}
\item The user is supposed to attend every meeting

\item A meeting should not take more time than expected.

\item If a user has a meeting in a specific location, he's supposed to be there at the end of the meeting, and the app will compute a route based on that information

\item The Metros and buses are supposed to be on time, travel times are calculated based on the expected route duration.
\end{itemize}

\subsubsection{General Assumptions}

\begin{itemize}

\item \textbf{[A1] Signup and Login}

Considering that the assignments provided do not say much generally about users without any reference to a possible signup or login, we assume that the registration is mandatory to create the first meeting, then every access to the app requires the login to manage each event saved. Please note that login parameters could be memorized to save and recover easily a user instance.

\item \textbf{[A2] Meeting management}

According to the requirements, we want to develop a system which allows the user to set his preferences with regards to the travels. Moreover, we decide that he can also cancel or anticipate/postpose an event, assuming a previous reschedule agreement among the participants. It goes without saying that an appointment can also be modified, this means that a user can change either the starting location or the arrival location, the hour, the date and the other details chosen during the creation of the meeting, always making the same rescheduling agreement assumption. 

\item \textbf{[A3] Warnings}

Our assumptions about the warning are the following: when the system generates a warning, the app allows the user to modify the related event that could be cancelled or delayed. In case of the user postposes the meeting, if he provided the email addresses of the other people involved in the appointment, Travlendar+ automatically will notify them that a change occurs. 

\item \textbf{[A4] Routes}

Concerning the routes, we decided to manage them in this way. 
The system generates different routes according to the user preferences, it will be the user itself to decide which itinerary fits better with him among the alternatives. 

\item \textbf{[A5] Preferences}

As far as the preferences are concerned, we decided that they belong to a user instead of a meeting. This means that a user cannot define different preferences for each meeting, while they are valid for every appointment.

\end{itemize}

 


\clearpage
\section{Specific requirements}

\subsection{External Interface Requirements}

\subsubsection{User Interfaces}

\subsubsection{Software Interfaces}

	Considering the domain of our application, we  decided to integrate in our project some software components to create an easier and more powerful product. 
	\\In order to provide an excellent navigation service, we thought to adopt the Google Maps API, to retrieve the user location through the Google Maps' servers and databases. 
	\\In addition, we noticed that the most complex computations which Travlendar+ should perform are those related to calculating the best route. This idea suggested us to use the APIs of several travel mean sharing services, such as Mobike or CarToGo, 
	to support this crucial phase. We are sure that these APIs are going to allow us to query the databases and to retrive precise information in the quickest and easiest way.
	\\The same reasoning is applicable to forecast, needed to avoid certain routes in case of particular wheater conditions, indeed wheater information from a specific provider are supposed to be rietrieved through its APIs. 
	\\Finally, APIs of public transport societies are required to retrieve real time information about buses,trains and taxis and to mantain the app up to date with relevance to all the related news, for instance about strikes. 
	


\subsection{Functional Requirements}
\begin{itemize}
		\item Logging in whitout being signed up is prevented. (f1)
		\item Logging in, being already logged in, is impossible. (f1)
		\item Signing up, being already signed up, is impossible.(f1)
		\item Creating a meeting with the same schedule of an existent one is not allowed. (f2)
		\item Scheduling a meeting during break pauses, setted in preferences, is prevented. (f2)
		\item it is impossible to generate an appointment in the past or in an invalid date. (f2)
		\item Meeting creation requires name,date,hour,location and a starting point to be defined properly.(f2)
		\item At least one mean of travel must be selected.(f3)
		\item Every lunch break must last at least 30 minutes.(f3)
		\item If rain or snow are in the forecast, travel by bus is preferred. (f6)
		\item In case of strikes, routes involving the related travel mean are not considered. (f6)
		\item Reminders must be setted before the scheduled time of the related events(f7)
	\end{itemize}


\subsection{Non functional requirements}
\begin{itemize}
\item Simple User Interface:
The user interface has to be as simple and intuitive as possible, the application should allow an average user to set up an account and start using the application understanding its functionality in no more than a dozen minutes.

\item Portability. The client has to be compatible to all the major hardware and software platform on the market, this is accomplished using the web application solution presented early.

\item Performance. The application should be able to calculate shortest paths very quickly in order to let the user choose the one that better fits his needs right after setting up the meeting.

\item Reliability. The system should be able to guarantee the service independently of the time, 24/24, 7/7. Thus, the used services should be always available, however, since this is not a critical application, brief unavailability could be acceptable.

\item Data integrity, consistency and availability. System data have to be always accessible. Hence the system should always provide a reliable access to them in normal condition. They also have to be duplicate in order to avoid data losses in case of system fault. 

\item Security. Hashed password should be stored in the database in order to guarantee a high level of privacy to the users. Sensible data such as meetings details will probably be stored locally on the user device, a possible encryption might be considered but it's not a priority
\end{itemize}

\subsection{The world and the machine}
The first model of the system to be presented is the model ``The world and the machine'' by M. Jackson and P. Zave. This model highlights the division between phenomena that happen entirely either in the world or in the machine, and those that are shared between the two of them.

immagine

\subsection{Scenarios}
\subsubsection{Scenario 1}

\subsubsection{Scenario 2}

\subsubsection{Scenario 3}

\clearpage
\subsection{Use Cases}
This section contains all the use cases initially described with the use cases UML model of the whole system.

\subsubsection{User Page use cases}
\begin{figure}[htp] 

\includegraphics[width=\textwidth]{usecases/png/userpage} 
\caption{Use cases relative to the user registration and authentication} 
\label{fig:userpage} 
\end{figure} 

\newpage
\subsubsection{Sign up}

\begin{table}[htp]

\begin{tabular}{r|p{7cm}}
\bf\large Name&\bf\large Sign up\\
\hline
\hline
\bf Actors&Guest\\
\hline
\bf Entry conditions&None\\
\hline
\bf Flow of events&
\begin{itemize}
\item The guest reaches the registration page containing the relative form
\item The guest fills up the form and clicks on "Sign up" to complete the process
\item The system redirects the user to his profile page and sends a confirmation email.
\end{itemize}
\\
\hline
\bf Exit conditions&The guest has successfully registered in the system. \\
\hline
\bf Exceptions&The guest left an empty field or typed
 something wrong an error message is displayed 
 and the user is asked to fill the form again.\\
\hline

\end{tabular}
\caption{This is my one big table} \label{tab:signup}
\end{table}


 \newpage
\subsubsection{Login}
\begin{table}[htp]

\begin{tabular}{r|p{7cm}}
\bf\large Name&\bf\large Login\\
\hline
\hline
\bf Actors&User\\
\hline
\bf Entry conditions&The user has already registered.\\
\hline
\bf Flow of events&
\begin{itemize}
\item The user reaches the login page containing the relative form
\item The user types the username and password in the login form and click on "Login" button.
\item The system redirects the user to the application homepage.
\end{itemize}
\\
\hline
\bf Exit conditions&The user has access to the application functionalities. \\
\hline
\bf Exceptions&Username and password didn't correspond or the username didn't exist ,an error message is displayed and the user is asked to fill the login form again.\\
\hline

\end{tabular}

\caption{Login Use Case table} \label{tab:login}
%\autoref{fig:userpage}
\end{table}
\newpage
\subsubsection{Password Recovery}
\begin{table}
\begin{tabular}{r|p{7cm}}
\bf\large Name&\bf\large Recover Password \\
\hline
\hline
\bf Actors&User\\
\hline
\bf Entry conditions&The user has already registered.\\
\hline
\bf Flow of events&
\begin{itemize}
\item The user reaches the login page containing the relative form
\item The user clicks on "Password recovery" button and is redirected to the password recovery page.
\item The user inserts his email and clicks on "reset password".
\item The system sends an email to the user with a link and instruction to reset the password.
\item The user chooses and types a new password and confirms.
\item The system redirects the user to the login page.
\end{itemize}
\\
\hline
\bf Exit conditions&The user has changed his password \\
\hline
\bf Exceptions&The inserted email doesn't match any user in the database, it is displayed an error message and the user is asked to retype a valid email.\\
\hline

\end{tabular}
\caption{This is my one big table} \label{tab:recoverpassword}
\end{table}


\newpage
\subsubsection{Schedule Management use cases}
\begin{figure}[htp] 
\includegraphics[width=\textwidth]{usecases/png/schedulemanagement} 
\caption{Main use cases showing the functionalities of Travlendar+ application relative to the meetings creation and management} 
\label{fig:schedulemanagement} 
\end{figure}

\newpage
\subsubsection{Meeting Creation}

\begin{figure} 
\begin{center}

\makebox[\textwidth]{%
\includegraphics[width=1.7\linewidth]{images/meetingcreation} 
}
\caption{Meeting creation sequence diagram} 
\label{fig:meetingcreation} 


\end{center}
\end{figure} 
\newpage
\subsubsection{Reminder Addition}
\begin{table}
\begin{tabular}{r|p{7cm}}
\bf\large Name&\bf\large Addition of a reminder\\
\hline
\hline
\bf Actors&User\\
\hline
\bf Entry conditions&The user is logged in and is on the page of a meeting\\
\hline
\bf Flow of events&
\begin{itemize}
\item The user clicks on "Add reminder" button and he is redirected to the page with the input form to add a reminder.

\item The user fills up the form with the type of reminder and the time he wants to be reminded of the upcoming meeting.

\item  The system adds the reminder and the user is redirected to the relative meeting page.

\end{itemize}
\\
\hline
\bf Exit conditions&The reminder is added to the meeting \\
\hline
\bf Exceptions&There exists already an identical reminder and it is not added to the meeting\\
\hline

\end{tabular}
\caption{This is my one big table} \label{tab:reminderaddition}
\end{table}
\newpage
\subsubsection{Warning Solving}
\begin{table}[htp]
\begin{tabular}{r|p{7cm}}

\bf\large Name&\bf\large Solving a warning\\
\hline
\hline
\bf Actors&User\\
\hline
\bf Entry conditions&The user is logged in and is in the page of a warning\\
\hline
\bf Flow of events&
\begin{itemize}
\item The user clicks on "Solve warning" button and he is redirected to a page that lets him
choose how to solve the conflict: the timing of two overlapping meetings can be changed, or one of the two meetings has to be canceled; 
\item  The user solves the conflict the way he wants and clicks on the button "Done".
\item  The system checks whether the conflict has been solved and the user is redirected to the warnings page.
\end{itemize}
\\
\hline
\bf Exit conditions&The warning has been solved and is deleted from the system and from the list of warnings in the corresponding page\\
\hline
\bf Exceptions&The warning was not solved after the user's modifications, the unresolved warning will still be present and a message stating that the conflict wasn't successfully solved is displayed. The user is redirected to the warning page.
\\
\hline

\end{tabular}
\caption{Warning solving Use Case table} 
\label{tab:warningsolving}
\end{table}


\newpage
\subsubsection{User Preferences use cases}
\begin{figure}[htp]
\includegraphics[width=\textwidth]{usecases/png/specifyuserpreferences} 
\caption{Use cases showing the user preferences and relative functonalities} 
\label{fig:specifyuserpreferences} 
\end{figure}

\newpage
\subsubsection{Means activation/deactivation}
\begin{table}[htp]
\begin{tabular}{r|p{7cm}}
\bf\large Name&\bf\large Activate/deactivate mean of transport\\
\hline
\hline
\bf Actors&User\\
\hline
\bf Entry conditions&The user is logged in and is in the user preferences page\\
\hline
\bf Flow of events&
\begin{itemize}
\item The user clicks on "Choose means of transport" button and he is redirected to a page containing a list of all possible means of transport;
\item  The user unflags all the means of transport he does not intend to use.
\item  The user clicks on "Done" button and is redirected to the user preferences page.
\end{itemize}
\\
\hline
\bf Exit conditions&The unflagged means of transport are removed from the possible means needed to compute a route\\
\hline
\bf Exceptions&The user unselected every mean of transport, clicking "Done" button has no effect and an error message stating that at lest one mean of trasport has to be flagged.
\\
\hline

\end{tabular}
\caption{This is my one big table} \label{tab:activatedeactivatemean}
\end{table}





\clearpage
\subsection{Activity Diagrams}

\subsubsection{Meeting creation process}
Whenever the user fills up the form relative to creation of a meeting and presses "Create meeting" the system does what is visually described in the diagram.
\\It checks for conflicts while asynchronously queries public transport providers, when the data gets back and the meeting is set as regular or is put in a warning with other meetings the meeting is created and the process is done.

\begin{figure}[htp] 

\includegraphics[width=\textwidth]{activitydiagrams/meetingcreationprocess} 
\caption{This diagram shows the activities carried out by the system in the meeting creation process} 
\label{fig:meetingcreationprocess} 
\end{figure} 


\clearpage
\subsection{State Charts}

\subsubsection{Meeting State Machine}
This State Machine was created with the purpose to identify the various states a meeting can be in and the show the transition events that modify its state.

\begin{figure}[htp] 

\includegraphics[width=\textwidth]{statecharts/meetingstatemachine} 
\caption{State Chart showing states of a meeting} 
\label{fig:meetingstatemachine} 
\end{figure} 

\newpage
\subsubsection{Basic UX State Chart}
We provide a simple and basic User Experience State Chart which highlights the different pages a user can find himself in (and the System redirects the user to, when an event happens). 
\\We believe this could be helpful to understand and visualize the entire application.

\begin{figure}[htp] 

\includegraphics[width=\textwidth]{statecharts/basicux} 
\caption{A basic User eXperience chart of the application} 
\label{fig:ux} 
\end{figure}

\clearpage
\subsection{Sequence Diagrams}


\clearpage
\subsection{Class Diagram}

\section{Alloy}
\subsection{Model}


\subsection{Result}
\begin{center}
	
	immagine dei risultati
\end{center}
\subsection{Worlds Generated}

\section{Effort Spent}

\end{document}
