The RASD Document is organized following a strict schema.
 In the first introductory section, we give a short description both of the goals and of the environment which our app has to deal with. Moreover, we explain some notes useful to understand and read the whole paper. 
The next part consists in an overall description of the software, presenting all the functions more in depth and showing the possible interactions between the user, the system and the world itself. 
The third part concerns the analysis of requirements, from the technological ones, through the functional, up to design details and constraints. 
Finally, we express the requirements through the Alloy model, which allows us to define the interactions, the functions and the constraints that characterize Travlendar+ using a formal language.
The document ends with a short note about the effort spent in producing it and at last you can also find useful references. 
